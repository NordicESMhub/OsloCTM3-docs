\documentclass[10pt,a4paper,twocolumn]{article}
\usepackage[latin1]{inputenc}
\usepackage[T1]{fontenc}
\usepackage[usenames]{color}
% For references layout
\usepackage{natbib}
% To make index at the end
\usepackage{makeidx}
\makeindex

% side margin
\oddsidemargin 0mm
\evensidemargin 0mm

% vertical dimension
%\topmargin 0pt 
\addtolength{\topmargin}{-27mm}
\addtolength{\textheight}{50mm} %{68pt}

% text width
\textwidth 160mm

% indent
\setlength{\parindent}{0mm}

% How many levels of subdirectories should be allowed
\setcounter{secnumdepth}{5}
%%%%%%%%%%%%%%%%%%%%%%%%%%%%%%%%%%%%%%%%%%%%%%%%%%%%%%%%%%%%%%%%%%%%%%

%%%%%%%%%%%%%%%%%%%%%%%%%%%%%%%%%%%%%%%%%%%%%%%%%%%%%%%%%%%%%%%%%%%%%%
% Setting up some short commands
% Line commands
\newcommand{\HRule}{\rule{\linewidth}{1mm}}
\newcommand{\VRule}{\rule{1mm}{60mm}}
\newcommand{\strek}{\rule{\linewidth}{0.1mm}}
\newcommand{\TynnRule}{\rule{\linewidth}{0.5mm}}
\newcommand{\MidtStrek}{\rule[0.8mm]{6mm}{0.1mm}}
% Bullet command
\newcommand{\bul}{$\bullet$\ }
% Tab command
\newcommand{\atab}{\hspace*{5mm}}

% Commands for names/text
\newcommand{\model}{Oslo CTM3}
\newcommand{\oldmodel}{Oslo CTM2}
\newcommand{\pmain}{p-main}
\newcommand{\ucimodel}{UCI CTM}
\newcommand{\titan}{Titan III}
\newcommand{\ocmail}{\texttt{oslo-ctm@geo.uio.no}}
%\newcommand{\ocweb}{\texttt{http://www.geo.uio.no/metos/osloctm/}}
\newcommand{\ocweb}{\texttt{http://folk.uio.no/asovde/osloctm3/}}
% Commands for e.g. files and directories
\newcommand{\pmainf}{{\it p-main.f}}
\newcommand{\makefile}{{\it Makefile}}
\newcommand{\params}{{\it params.h}}
\newcommand{\paramsF}{{\it params.F}}
\newcommand{\tracerlist}{{\it chch\_oc.d}}
\newcommand{\scavlist}{{\it scavng55\_oc.d}}
\newcommand{\maininput}{{\it LxxTyy.inp}}
\newcommand{\diaginput}{{\it LxxTyy.inp}}
\newcommand{\tracerinput}{{\it Ltracer\_oc.inp}}
\newcommand{\emisfile}{{\it Ltracer\_emis.inp}}
\newcommand{\avgsavfile}{{\it avgsavDDDDD}}
\newcommand{\oslopath}{{\it OSLO}}
\newcommand{\emispath}{{\it /data2/geo/EMIS}}
\newcommand{\qcodepath}{{\it UCI\_Q56D}}
\newcommand{\inptrctm}{{\it inptr.ctm}}
\newcommand{\inputctm}{{\it input.ctm}}
\newcommand{\pwind}{{\it p-wind\_ec.f}}
\newcommand{\diagnosticsctm}{{\it diagnostics.ctm}}
\newcommand{\qcodedir}{{\it UCI\_Q56D}}

% New environments
\newenvironment{itemize2} {
\begin{itemize}\addtolength{\itemsep}{-0.5\baselineskip}\vspace*{-0.8\baselineskip}
}{\end{itemize} }
\newenvironment{description2} {
\begin{description}\addtolength{\itemsep}{-0.6\baselineskip}
}{\end{description} }
%%%%%%%%%%%%%%%%%%%%%%%%%%%%%%%%%%%%%%%%%%%%%%%%%%%%%%%%%%%%%%%%%%%%%%
%%%%%%%%%%%%%%%%%%%%%%%%%%%%%%%%%%%%%%%%%%%%%%%%%%%%%%%%%%%%%%%%%%%%%%
%%%%%%%%%%%%%%%%%%%%%%%%%%%%%%%%%%%%%%%%%%%%%%%%%%%%%%%%%%%%%%%%%%%%%%
% Begin document
%%%%%%%%%%%%%%%%%%%%%%%%%%%%%%%%%%%%%%%%%%%%%%%%%%%%%%%%%%%%%%%%%%%%%%
%%%%%%%%%%%%%%%%%%%%%%%%%%%%%%%%%%%%%%%%%%%%%%%%%%%%%%%%%%%%%%%%%%%%%%
%%%%%%%%%%%%%%%%%%%%%%%%%%%%%%%%%%%%%%%%%%%%%%%%%%%%%%%%%%%%%%%%%%%%%%
\begin{document}
%%%%%%%%%%%%%%%%%%%%%%%%%%%%%%%%%%%%%%%%%%%%%%%%%%%%%%%%%%%%%%%%%%%%%%

% Set space between paragraphs (to avoid using \\)
\setlength{\parskip}{3mm}


%%%%%%%%%%%%%%%%%%%%%%%%%%%%%%%%%%%%%%%%%%%%%%%%%%%%%%%%%%%%%%%%%%%%%%
% Title
%%%%%%%%%%%%%%%%%%%%%%%%%%%%%%%%%%%%%%%%%%%%%%%%%%%%%%%%%%%%%%%%%%%%%%
% September 2008, Amund Sovde
\twocolumn[\flushright
{\bf \huge \model}\\
\vspace*{3mm}
{\bf \Large SVN instructions}\\
\vspace*{3mm}
{\bf \large Ole Amund S{\o}vde}\\
{\bf Department of Geosciences}\\
{\bf University of Oslo}\\
{\bf \today}\\
\vspace*{10mm}
\bigskip
]
\small
%%%%%%%%%%%%%%%%%%%%%%%%%%%%%%%%%%%%%%%%%%%%%%%%%%%%%%%%%%%%%%%%%%%%%%%
%%%%%%%%%%%%%%%%%%%%%%%%%%%%%%%%%%%%%%%%%%%%%%%%%%%%%%%%%%%%%%%%%%%%%




%%%%%%%%%%%%%%%%%%%%%%%%%%%%%%%%%%%%%%%%%%%%%%%%%%%%%%%%%%%%%%%%%%%%%
%%%%%%%%%%%%%%%%%%%%%%%%%%%%%%%%%%%%%%%%%%%%%%%%%%%%%%%%%%%%%%%%%%%%%%

%%%%%%%%%%%%%%%%%%%%%%%%%%%%%%%%%%%%%%%%%%%%%%%%%%%%%%%%%%%%%%%%%%%%%%
% SVN -- Subversion -- #SVN
%%%%%%%%%%%%%%%%%%%%%%%%%%%%%%%%%%%%%%%%%%%%%%%%%%%%%%%%%%%%%%%%%%%%%%
% July 2009, Amund Sovde
\section{SVN -- Subversion}
\label{app:svn}
The \model\ is available through the UiO central subversion system
(SVN)\footnote{http://www2.usit.uio.no/it/subversion/}\index{SVN}.
To access the code you have to be member of the group
\verb#gf-ozone#.

SVN is a version control system, which replaces the old CVS version of
\oldmodel. A good introduction is the SVN book, available at
{\it http://svnbook.red-bean.com/}\index{SVN!manual}.

To use SVN you need to have SVN installed/loaded. If it is available
as a module, you can load it with
\begin{verbatim}
module load svn
\end{verbatim}

Important SVN commands are \verb#checkout#, \verb#add#,
\verb#delete#/\verb#remove#, \verb#commit#, \verb#status#,
\verb#diff#, \verb#resolve#.



% Repository structure
\subsection{Repository structure}
\label{app:svn_repository}\index{get started!repository structure}
Only the \model\ code, i.e. the \verb#trunk# of SVN, is described
in this manual. But there are other tools available through SVN. The
more important one is \verb#utilities#, and when you are experienced
enough to use branches and tags, you have separate directories for them.
\begin{verbatim}
  trunk/       (Oslo CTM3 code)
    MANUAL/    (this manual)
    OSLO/      (Oslo chemistry code)
      DEAD_COLUMN/ (Dust code)
      DUMMIES/     (Dummy routines)
      ROUTINES_NOT_USED/   (Old routines)
    UCI_Q56D/  (original UCI code)
    tables/    (tables)
    tmp/       (for pathscale compilation)
  branches/
  tags/
  utilities/
    fortran/
    idl/
    matlab/
  ctm2/            (old CTM2 code)
\end{verbatim}


% Checkout
\subsection{Checkout}\index{SVN!checkout}
\index{code!where is it?}\index{source code!where is it?}
The \model\ code is located in a {\it repository}, and you get it by
typing
{\small
\begin{verbatim}
svn checkout \
  svn+ssh://svn.uio.no/svnroot/osloctm3/trunk \
  <name of dir to put the code>
\end{verbatim}
}
The code is now available in your working directory. In the {\it .svn}
directory the original revision files are also available.

If you don't specify the name of directory where the code will be
copied to, it will be named {\it trunk}.

Similarly you can fetch the \verb#utilities# or any other directory,
just swap ``trunk'' with the directory you want. For example, if you
only want this manual, use \verb#trunk/MANUAL#.


% Status
\subsection{Status}\index{SVN!status}
SVN can check your working directory files against the original
revision of the repository. {\it It does not check the repository
itself!} The status tells you whether the file has changed or not.
{\small
\begin{verbatim}
svn status <file or path>
\end{verbatim}
}
If a file is specified, only that file is checked. If a directory is
specified (no specified path or file means current directory) all
files in the directory are checked.

The status returns a letter for each file specified:
\begin{itemize2}
  \item ? File does not exist in repository.
  \item A File is scheduled for addition.
  \item D File is scheduled for removal.
  \item C File is in conflict with repository.
  \item M File is locally modified.
\end{itemize2}

{\bf Important}\\
The current version (i.e. the latest repositiory trunk) may have
changed even though the file status reports the file to be unchanged.
Adding an option to the command allows you to see which
files that have newer versions in the repository (marked by \verb#*#):
{\small
\begin{verbatim}
svn status -u <file or path>
\end{verbatim}
You can also add a verbose option \verb#-v#.
}


% Conflicts & resolve
\subsection{Conflicts \& resolve}\index{SVN!conflicts}
\index{SVN!conflicts!resolving}\index{SVN!resolve}
Conflicts occur when you try to update your files, and SVN does not
know how to implement your changes into the repository.
This must then be {\it resolved} before you can
commit the changes. Resolving can be carried out immediately or be
postponed, which is usually the best way.

When getting a conflict for a file, several files are generated, their
file names being the original file name + an extension.
\begin{itemize}
  \item{.mine} Your working copy based on revision XX.
  \item{.rXX} Revision which your working copy was based on.
  \item{.rYY} The new file in the repository.
\end{itemize}
If you use svn older than version 1.5, you solve this by doing changes
to the file and then type:
{\small
\begin{verbatim}
svn resolved <file in conflict>
\end{verbatim}
}

For version 1.5 and newer, the command is resolve, and it takes
several options. You can decide to use your version
{\small
\begin{verbatim}
svn resolve --accept mine-full \
  <file in conflict>
\end{verbatim}
}
This can also be done by manually correcting the conflicts in the
file, followed by
{\small
\begin{verbatim}
svn resolve --accept working \
  <file in conflict>
\end{verbatim}
}
Other possibilities are to accept the repository version (no changes
will be committed)
{\small
\begin{verbatim}
svn resolve --accept theirs-full \
  <file in conflict>
\end{verbatim}
}
For resolving conflicts, see Chapter 2 in the SVN book.


% Adding a file
\subsection{Adding a file}\index{SVN!add file}
When adding a new file you must first
{\small
\begin{verbatim}
svn add <the new file>
\end{verbatim}
}
which schedules the file for the repository, then you can commit the
file:
{\small
\begin{verbatim}
svn commit <the new file>
\end{verbatim}
}


% Deleting a file
\subsection{Deleting a file}\index{SVN!delete file}
If you want to delete a file, simply use the \verb#delete# or
\verb#remove# commands, e.g.
{\small
\begin{verbatim}
svn delete <remove-file>
\end{verbatim}
}
which schedules the file for removal, and also removes your local
copy, so it is ready for committing:
{\small
\begin{verbatim}
svn commit <remove-file>
\end{verbatim}
}

You can delete directories in the same way. In that case, however, the
local directory is not removed until you commit the changes.


% Commiting changes
\subsection{Commiting changes}\index{SVN!commit changes}
Not everybody should be allowed to commit changes. A moderator should
approve your work before it is committed.

When it has been agreed that your modified files are to be included to
the repository, you first need to make sure the file is otherwise
updated:
{\small
\begin{verbatim}
svn update
\end{verbatim}
}
Then you can commit the changes, unless there are conflicts;
{\small
\begin{verbatim}
svn commit -m ``msg''
\end{verbatim}
}
where ``msg'' is a message to tag the changes. Instead of a message
you can send in a file where you describe the changes. This is done by
\verb#-F <filename># instead of \verb#-m ``msg''#.

If you only want to commit some of the changed files, use
{\small
\begin{verbatim}
svn commit <files to commit> -m ``msg''
\end{verbatim}
}



% Diff
\subsection{Diff}\index{SVN!diff}
Diff will give you the differences between your working copy and the
revision of the repository it is updated to, i.e. what is located in
the {\it .svn} directory. It does not check the repository!

If you want to compare with other revisions, use the option
\verb#-r revision_number#; this will check the repository. The
\verb#revision number# can take several forms, which you find in
Chapter 3 of the SVN book. The useful revision is usually the latest,
and the best solution for this is probably:
\begin{verbatim}
  svn diff -r {date} <file>
\end{verbatim}
where \verb#date# may be e.g. \verb#YYYY-MM-DD# or \verb#HH:MM#.

See Chapter 2 in the SVN book for more.


% Log
\subsection{Log}\index{SVN!log}
The changes you specify with \verb#-m# when you commit will be stored
in a log. You access the log by
\begin{verbatim}
  svn log
\end{verbatim}
to get all changes reported, or by 
\begin{verbatim}
  svn log <file>
\end{verbatim}
to get changes for a specific file.


% Help
\subsection{Help}\index{SVN!help}
You find out more about the SVN commands by typing
\begin{verbatim}
  svn help <command>
\end{verbatim}


% Braching
\subsection{Branching}\index{SVN!branching}
When you start working on a large modification to the \model, it is
wise to create a {\it branch} containing your code. SVN will keep
track of your files as before, but it will not affect the trunk --
only the branch.
Later, when the branch is working well, the branch may be merged into
the trunk.

Read Chapter 4 in the SVN book to find out more on this.

%%%%%%%%%%%%%%%%%%%%%%%%%%%%%%%%%%%%%%%%%%%%%%%%%%%%%%%%%%%%%%%%%%%%%%
%%%%%%%%%%%%%%%%%%%%%%%%%%%%%%%%%%%%%%%%%%%%%%%%%%%%%%%%%%%%%%%%%%%%%%


%%%%%%%%%%%%%%%%%%%%%%%%%%%%%%%%%%%%%%%%%%%%%%%%%%%%%%%%%%%%%%%%%%%%%%
\end{document}
%%%%%%%%%%%%%%%%%%%%%%%%%%%%%%%%%%%%%%%%%%%%%%%%%%%%%%%%%%%%%%%%%%%%%%


