% Speeding up the sub-stepping loop
\subsection{Speeding up the sub-stepping loop}
\label{app:substepping_faster}\index{sub-stepping loop!faster?}
The original solution by UCI is to loop through \verb#NLCM# and check
whether advection or chemistry and boundary layer mixing should be
done.

If it is necessary to have e.g. \verb#NRCH=4#, the following approach
should be considered to save the number of times going in and out of
parallel blocks. It is slightly different from the LCM approach, as
will be explained.

Realizing that advection controls the choice of parallelization, let
the sub-stepping \verb#NSUB# loop through the global advection steps
(\verb#NADV#). For each \verb#NSUB# step the number of {\it chemical}
sub-steps per advection step (\verb#NSSPADV#) can be calculated
(number of chemistry and boundary layer mixing steps during this
\verb#NSUB#).
This is outlined in Table \ref{tbl:nsubmsub}.
\begin{table}[t!]
  \caption{Overview of the operator split loop with two sub-stepping
    loops.}
  \label{tbl:nsubmsub}
\begin{minipage}{\linewidth}
\strek
\vspace{-1mm}
\begin{verbatim}
   !// OPERATOR SPLIT LOOP
   do NOPS = 1,NROPSM

     do NSUB=1,NADV

       <calculate NSSPADV>
       <into parallel IJ-block>
       do MSUB=1,NSSPADV
         <boundary layer mixing>
         <chemistry>
       end do
       <convection>
       <w-advection>
       <end parallel IJ-block>
       <into parallel layers>
       <uv-advection>
       <end parallel layers>

     end do

   end do
\end{verbatim}
\vspace{-4mm}\strek
\end{minipage}
\end{table}

The processes carried out per \verb#NSUB# will then have time step
\verb#DTADV# while the ones in \verb#MSUB#\index{MSUB} have time step
\verb#DTCHM#.

For Example 1 in Table \ref{tbl:substepping}, \verb#NSUB# loops once,
but \verb#MSUB# loops four times.
An obvious difference from the UCI approach is that now the sequence
is C/C/C/C/A instead of C/A/C/C/C, since the 
In other words are the sequence shifted slightly, and this should be
of minor importance.

In Example 2, the sub-steps are
\begin{verbatim}
  NSUB  MSUB
   1     2
   2     1
   3     1
\end{verbatim}
so the new sequence is C/C/A/C/A/C/A, where the UCI method gives
C/A/C/A/C/A/C. Again we see a shift, which should be of minor
importance.

As a rather unrealistic example (at least for T42 horizontal
resolution), \verb#NADV=5# and \verb#NRCHEM=4# would result in
\begin{verbatim}
  NSUB  MSUB
   1     1
   2     1
   3     1
   4     1
   5     0
\end{verbatim}
