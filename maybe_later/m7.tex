%\index{M7!history}
The documented aerosol modules (Sulphate, BCOC, DUST, SALT) produce
aerosols independent of each other. Nitrate is somewhat special since
it depends on SALT and sulphate, and sulphate depends on the
tropospheric chemistry.

M7 tries to put the aerosol packages together. The inclusion of M7
builds on this approach in the following way:
\begin{enumerate}
  \item All the separate modules are left untouched.
  \item If M7 is activated the emissions of the components which 
    are concerned by M7 are transferred to a new array and the 
    original emission arrays become irrelevant.
  \item All the M7 tracers have their own loss terms depending
    on the microphysics.
\end{enumerate}

You turn on the M7 application by setting \verb#M7 := Y# in
\makefile. Running the aerosol modules separately is of coarse good if
you want to save CPU time and only want to run 8 carbon tracers
instead of full chemistry/sulfate/M7 run.


% Physics
\subsection{Physics}
\index{M7!physics}
M7 is a microphysical model. This means that it's purpose is to
calculate interaction between different types of aerosols. The output
is the size of the aerosols and their mixing state (i.e. water soluble
or insoluble). M7 splits the aerosol size distribution into the
following size distributed modes
\begin{itemize2}
  \item Nucleation mode, water soluble, size: $\sim 1 nm$
  \item Aitken mode, water soluble,  size: $\sim 0.01 {\mu}m$
  \item Accumulation mode, water soluble, size: $\sim 0.1 {\mu}m$
  \item Coarse mode, water soluble, size: $\sim 1 {\mu}m$
  \item Aitken mode, insoluble,  size: $\sim 0.01 {\mu}m$
  \item Accumulation insoluble, size: $\sim 0.1 {\mu}m$
  \item Coarse mode, insoluble, size: $\sim 1 {\mu}m$
\end{itemize2}

{\bf Important}\\
{\bf M7 does not treat semivolatile aerosol species} like nitrate or
secondary organics. This is both a good thing and a bad thing. It is
good because it greatly  reduces the complexity of the problem, and it
is bad because.. well... you can not calculate the effect these
tracers would have on aerosol microphysics.

% M7 tracers
\subsection{M7 tracers}
\index{M7!tracers}
The M7 tracers are listed in Table \ref{tbl:m7tracers}.

\begin{table*}
\caption{Tracers included in the M7 model. All the aerosol species
  (250--274) have lifetimes of days-weeks and they all need to be
  transported. Water species (275--278) are in equilibrium (see below)
  and should not be transported. Soluble species (except water) are
  washed out in convection. There are 17 soluble species in M7.}
\label{tbl:m7tracers}
\begin{minipage}{\linewidth}
\begin{tabular}{|c|l|l|c|c|}
\hline
Nr  & Component name & Remarks                 & Soluble   &  Transport \\
\hline \hline
80  & SO4GAS     &  Gas phase sulf. acid       &  N/A (No) & No\\
\hline
250 & SO4NUC     &    nucleation mode sulphate &  Yes      &  Yes\\
\hline
251 & SO4AT      &    aitken mode sulphate     &  Yes      &  Yes\\
\hline
252 & SO4AC      &    accumulation mode sulphate &  Yes    &  Yes\\
\hline
253 & SO4CRS     &    coarse mode sulphate  &  Yes       &  Yes\\
\hline
254 & BCKS       &  aitken mode soluble BC  &  Yes       &  Yes\\
\hline
255 & BCAS       &    accumulation mode BC  &  Yes       &  Yes\\
\hline
256 & BCCS       &    coarse mode BC        &  Yes       &  Yes\\
\hline
257 & BCKI       &    aitken mode BC        &  No        &  Yes\\
\hline
258 & OCKS       &    aitken mode OC        &  Yes       &  Yes\\
\hline
259 & OCAS       &    accumulation mode OC  &  Yes       &  Yes\\
\hline
260 & OCCS       &    coarse mode OC        &  Yes       &  Yes\\
\hline
261 & OCKI       &    aitken mode OC        &  No        &  Yes\\
\hline
262 & SSAS       & accumulation mode Seasalt &  Yes      &  Yes\\
\hline
263 & SSCS       &  coarse mode seasalt      &  Yes      &  Yes\\
\hline
264 & DUAS       &    accumulation mode dust  &  Yes     &  Yes\\
\hline
265 & DUCS       &    coarse mode dust        &  Yes     &  Yes\\
\hline
266 & DUAI       &    accumulation mode dust  &  No      &  Yes\\
\hline
267 & DUCI       &   coarse mode dust        &  No       &  Yes\\
\hline
268 & NNUC       &    Nucleation mode number &  Yes      &  Yes\\
\hline
269 & NKS        &    Aitken mode number      &  Yes     &  Yes\\
\hline
270 & NAS        &    Accumulation mode number  &  Yes   &  Yes\\
\hline 
271 & NCS        &   coarse mode number        &  Yes    &  Yes\\
\hline
272 & NKI        &   aikten mode number      &  No       &  Yes\\
\hline
273 & NAI        &    accumulation mode number  &  No    &  Yes\\
\hline
274 & NCI        &   Coarse mode number         & No     &  Yes\\
\hline
275 & H2ONUC     &   nucleation mode water      &  Yes   &  No\\
\hline
276 & H2OKS      &  aitken mode water           &  Yes   &  No \\
\hline
277 & H2OAS      &  accumulation mode water     & Yes    &  No\\
\hline
278 & H2OCS      &  coarse mode water           & Yes    &  No \\
\hline
\end{tabular}
\end{minipage}
\end{table*}

Thus, when running M7, you need 51 tracers from tropospheric
chemistry, 7 from sulfate and one extra (\verb#SO4GAS#), and further
you need the 8 BC/OC species, which are not transported or washed
out. The tracers washed out in convection are all the
water-soluble. Table \ref{tbl:tracerm7} shows in more detail the
tracers need for running M7.

\begin{table}
\caption{Explanation of the number of tracers needed for running
  M7. Note that several tracers are dummy tracers (the 8 of BCOC, the
  2 of dust, the 2 of salt). These tracers are handled by the 25 M7
  tracers when running M7. The sulphate species are converted
  immediately to M7 species. The number in parenthesis for
  Transport indicates the  number of tracers if you exclude the
  tracers labeled \textcolor{red}{\it Maybe} in Table
  \ref{tbl:tropcomp}.}
 \label{tbl:tracerm7}
\begin{minipage}{\linewidth}
\begin{tabular}{|l|c|c|c|}
\hline
Application      & Tracers  & Transport  & Washout \\
\hline
TROPCHEM         &  51      &  40 (35)   &  8 \\
\hline
SALT             &  2\footnote{dummy tracers} &  0   &  0 \\
\hline
DUST             &  2\footnote{dummy tracers} &  0   &  0 \\
\hline
SULPHATE\footnote{no transport or washout of old $SO_4$} 
                 & 7        &  4          &  3 \\  
\hline
SO4GAS           &  1       &  0          &  0 \\
\hline
BCOC             &  8\footnote{dummy tracers}  &  0   &  0 \\
\hline
M7               &  25      &  25         &  17 \\
\hline
M7(water)        &  4       &  0          &  0 \\
\hline
\hline
SUM              & 100      &  69 (64)    &  28 \\
\hline
\end{tabular}
\end{minipage}
\end{table}


\subsection{Technical information}
\index{M7!technical information}

The M7 model is activated by the logical switch LM7. This logical
variable currently has to be set in the main program. When this
logical switch is activated, the emissions calculated for the
M7-relevant components are removed from the \verb#POLLX# array and put
in the \verb#PRODM7# array. 

The idea is that the old aerosols tracers still exist in the
model. However, they are in a way irrelevant if M7 is used and there
is no need for transporting them.


\subsection{Interaction with other parts of aerosol codes}
\index{M7!aerosol code interaction}

\subsubsection{Carbon particles}
\index{M7!carbon particles}
The carbon particles in their old form should not be transported if M7
is used. However, they should be included as tracers in the \inptrctm\
file.

If the logical LM7 is recognized by the model, the emissions of BC
and OC are put into the \verb#prodm7# array and enters the M7 tracer
list. They are not put into the places they have as BC/OC tracers as
described in section \ref{sxn:bcoc}.

\subsubsection{Sulphate model}
\index{M7!sulphate model}
M7 needs explicit input about how much sulphate is formed in gas phase
and how much is formed in the aqueous phase. Therefore, another
tracer, number 80, \verb#SO4gas# is used when M7 is used.
The tracer which represents sulphate in the sulphate application (see
chapter \ref{sxn:sulphate}) which is tracer number 73 is only
representing aqueous phase sulphate when M7 is used.

There is no need for transporting the sulphate tracers since the mass
produced in the sulphate tracers are moved to the M7-tracers in the
same timestep. These tracers are later the ones being transported and
being lost by dry- and wet deposition.

\subsubsection{Dust and Seasalt models}
\index{M7!dust and seasalt models}

The dust and seasalt models are activated when it finds tracers
named \verb#SALTXX# or \verb#DUSTXX# (where \verb#XX# is bin number)
in the standard version (see more description in section
\ref{sxn:dust} and \ref{sxn:salt}. To not re-organize the code too
much, this feature is kept, and you still need tracers called
\verb#DUST01# and \verb#SALT01# to be able to run these modules.

However, when running M7, the standard versions of these codes are
no longer relevant. Therefore, you need 2 dust species and 2 salt
species to activate the model. These species have to be listed in
\inptrctm\ {\bf but they do not need to be transported}.

\subsubsection{Salt model}
\label{sxn:salt2m7}\index{M7!seasalt model}

The logical variable LM7 is sent directly into the salt model.
If the salt model recognizes this variable as TRUE, it will
not care about fall speeds and dry deposition, but only output
the emissions into the \verb#prodm7# array.

In the case of LM7 being true in the seasalt application, a production
routine giving modal emissions are being activated. This is different
from the standard model in the way that in the standard model,
the production routines distributes the production to size bins.

\subsubsection{Dust model}
\index{M7!dust model}

When using M7, the whole DUST model (DEAD) is still run. This is 
strictly not necessary since M7 does not really care about what
happens in DEAD except for one thing: It wants the emissions.

It would have been possible to send the logical LM7 to the dust model 
and only output the emissions if M7 is used, and not call the
deposition routines in a similar way to what is done for the salt
model (section \ref{sxn:salt2m7}. However: In the case of dust, 
the complete call to DEAD  kept because of the following reasons
\begin{itemize}
  \item The dust code is more complicated than the salt code and
    sending around the logical LM7 in it could be messy.
  \item DEAD is not developed at UiO, and if anyone wants to upgrade
    to a more recent version of DEAD, it would be easier not having to
    re-code it using the variable LM7 also.
\end{itemize}

The code to DEAD will update the concentrations of the tracer
\verb#DUST01# which should not transported in CTM2. However, DEAD also
puts the emissions in a 2D field which can be accessed with the
subroutine \verb#dustfield2main#.

All that is needed to include dust in M7 is the call to
\verb#dustfield2main# and fetch the emissions. These are then
processed in the \verb#prodm7# variable.

